\documentclass[11pt,a4paper]{article}
\usepackage[slovene]{babel}
\usepackage[utf8]{inputenc}
\usepackage{graphicx}
\usepackage{float}
\usepackage{amsthm,amsmath,amssymb}
\usepackage{mathtools}
\usepackage[makeroom]{cancel}
\usepackage{ulem}
\usepackage{listings}
\usepackage[noend]{algpseudocode}
\usepackage{nameref}
\usepackage{enumitem}

\normalem

\DeclareMathOperator{\Ex}{E}
\DeclareMathOperator{\Pro}{P}
\DeclarePairedDelimiter{\floor}{\lfloor}{\rfloor}
\DeclarePairedDelimiter{\ceil}{\lceil}{\rceil}

% Ukazi za izreke, trditve, ipd.
{
    \theoremstyle{plain}
    \newtheorem{theorem}{Izrek}[section]
    \newtheorem{lemma}[theorem]{Lema}
    \newtheorem{corollary}[theorem]{Posledica}
    \newtheorem{proposition}[theorem]{Trditev}
    \newtheorem{claim}[theorem]{Trditev}
    \newtheorem{conjecture}[theorem]{Hipoteza}
    \newtheorem{problem}[theorem]{Problem}
}

{
    \theoremstyle{definition}
    \newtheorem{definition}[theorem]{Definicija}
    \newtheorem{remark}[theorem]{Opomba}
    \newtheorem{observation}[theorem]{Opazka}
    \newtheorem{example}[theorem]{Zgled}
}

\lstset{
basicstyle=\ttfamily,
keywordstyle=\ttfamily,
columns=fixed,
showstringspaces=false,
mathescape
}

\title{2. naloga \\ \normalsize Aktualno raziskovalno področje IIF -- Inženiring algoritmov}
\author{Sven Cerk\\27152037}


\begin{document}
\maketitle

\section{Štetje inverzij}

\subsection{} 

\begin{definition}[Inverzija]
Naj bo $a_1, a_2, \ldots, a_n$ zaporedje celih števil. \emph{Inverzija} je par indeksov $(i,j)$ ($i,j \in \{1,2,\ldots,n\}$) za katera velja $i < j$ in $a_i > a_j$.
\end{definition}

\begin{definition}[Število inverzij]
\emph{Število inverzij} je moč množice vseh inverzij v zaporedju $(a_i)_{i = 1,\ldots,n}$
\[
\left| \{ (i,j) \mid \text{$(i,j)$ inverzija v zaporedju $(a_i)_{i = 1,\ldots,n}$} \} \right|
\]
\end{definition}

Naloga je definirana kot
\begin{lstlisting}[escapechar=~]
~\normalfont\textbf{Vhod}: $(a_1, a_2, \ldots a_n)$ zaporedje celih števil~
~\normalfont\textbf{Izhod}: Število inverzij v zaporedju $(a_1, a_2, \ldots a_n)$~
\end{lstlisting}

\subsection{}

Najmanjše število inverzij je v zaporedju urejenem naraščajoče --- v tem primeru je število inverzij 0. Največje število inverzij je v zaporedju urejenem padajoče --- $(n-1) + (n-2) + (n-3) + \cdots + 2 + 1 + 0 = \frac{n (n-1)}{2} = \frac{n^2 - n}{2}$.

\subsection{}

Število različnih vhodov:
\begin{enumerate}[label=\alph*)]
\item $a_i \in \mathbb{Z}_{32b}$ : $2^{32n}$
\item $a_i \in \mathbb{Z}_{32b}$, $a_i \neq a_j$ : $\frac{(2^{32})!}{(2^{32} - n)!}$
\item $a_i \in \mathbb{Z}_{n}$ : $n^n$
\item $a_i \in \mathbb{Z}_{n}$, $a_i \neq a_j$ : $n!$
\end{enumerate}

\subsection{}

\begin{enumerate}[label=\alph*)]
\item brez inverzij: $\frac{1}{n!}$
\item na zadnjem mestu 0: $\frac{(n-1)!}{n!}$
\item na prvem 0 in zadnjem $(n-1)$: $\frac{(n-2)!}{n!}$
\end{enumerate}

\subsection{}

\begin{enumerate}[label=\alph*)]
\item
Glavna ideja za algoritem grobe sile je, da preverimo vse možne pare, če predstavljajo inverzijo ali ne ter štejemo inverzije.

\item
\begin{algorithmic}
\State $count \gets 0$
\For {$i \gets 1 \ldots (n-1)$}
    \For {$j \gets (i+1) \ldots n$}
        \If {$a_i > a_j$}
            \State $count \gets count + 1$
        \EndIf
    \EndFor
\EndFor
\Return {$count$}
\end{algorithmic}

\item
Časovna zahtevnost algoritma je
\begin{align*}
T(n) &\sim \sum_{i=0}^{n-2} \left( \sum_{j=i+1}^{n-1} 1 \right) \\
     &\sim \sum_{i=0}^{n-2} \left( n - 1 - i \right) \\
     &\sim \sum_{i=0}^{n-2} \left( n-1 \right) - \sum_{i=0}^{n-2} i \\
     &\sim (n-1)^2 - \frac{(n-1) (n-2)}{2} \\
     &\sim n^2 - \frac{n^2}{2} \\
     &\sim \frac{\ n^2}{2}
\end{align*}

\end{enumerate}

\subsection{}

\end{document}
